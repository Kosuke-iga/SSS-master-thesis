\documentclass{standalone} 
\usepackage{tikz}
\usetikzlibrary{positioning, shapes, automata, arrows}
\usetikzlibrary{shapes.geometric}
\usepackage{pgfplots}
\usepackage{latexsym}
\usepackage{algorithm}
\usepackage{algorithmic}
% \pgfplotsset{compat=latest}
\usepackage{amsmath}
\usepackage[varg]{txfonts}
\usepackage{pgfplotstable}
\usetikzlibrary{patterns}
\def\minYear{1987}
\def\maxYear{2014}
\def\xmin{\minYear-1900-0.5}
\def\xmax{\maxYear-1900+0.5}
\pgfplotsset{compat=1.13}  % PGFPlots 1.13の機能セットを使用することを宣言

\begin{document}
\pgfplotsset{
	select row/.style={
		x filter/.code={\ifnum\coordindex=#1\else\def\pgfmathresult{}\fi}
	}
}
\pgfplotstableread[col sep=comma,header=false]{
	0-0.1,13
	0.1-0.2,21
	0.2-0.3,712
	0.3-0.4,3425
	0.4-0.5,961
	0.5-0.6,206
	0.6-0.7,37
	0.7-0.8,22
	0.8-0.9,17
	0.9-1,86
		% 2013/14(YTD),97
}\mal

\pgfplotstableread[col sep=comma,header=false]{
	0-0.1,747
	0.1-0.2,571
	0.2-0.3,1504
	0.3-0.4,719
	0.4-0.5,12
	0.5-0.6,5
	0.6-0.7,4
	0.7-0.8,2
	0.8-0.9,0
	0.9-1,0
		% 2013/14(YTD),97
}\leg

\begin{tikzpicture}[scale=1.0]
	\begin{axis}[
      ybar, bar shift=0pt,
      enlarge y limits=0.0,
      ytick={0,500, 1000, 1500, 2000, 2500, 3000, 3500}, 
			% xmin=0,
			xtick={0,...,9},
			% xtick={0.5,...,9}, 
			xticklabels from table={\mal}{0},
      ymajorgrids = true,
			bar width=3mm, 
			width=11cm, height=7cm, 
			xlabel={Score},
			ylabel={The number of PDF},
      x tick label style={font=\footnotesize,rotate=45, anchor=east},
    nodes near coords align={horizontal}
  ]
		\pgfplotsinvokeforeach{0,...,9}{
      \addplot [gray, pattern=north west lines, shift={(-1.5mm,0)}, legend image post style={xshift=1.5mm}] table [x expr=\coordindex, select row=#1] {\leg}; 
      \addplot [gray, fill, shift={(1.5mm,0)}, legend image post style={xshift=-1.5mm}] table [x expr=\coordindex, select row=#1] {\mal}; 
		}
    \addlegendentry{Legitimate PDF}; 
    \addlegendentry{Malicious PDF}; 
		% \node[
				% pin={[pin distance=1cm, pin edge={<-,>=stealth'},
					% shift={(3.2cm,-1.0cm)}]
      % Malicious features}] at (axis cs:3.3,2000) {};
	\end{axis}
\end{tikzpicture}

\end{document}
