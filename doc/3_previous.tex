\chapter{Previous Scheme}\markright{Previous Scheme}\label{sec:previous_scheme}
\section{Overview of the previous scheme} 
In order to deal with encrypted data, the traffic pattern based features are extracted without deep packet inspection in the previous scheme \cite{garg2017network}.
The main idea of the previous scheme is that there exists the difference of network traffic patterns between benign apps and malwares.
% Malwares need to transmit sensitive information to attackers most of the time the servers of attackers are offline or shutdown to escape detection.
Malwares more frequently use HTTP-POST/GET methods for transmission of sensitive data at short intervals than benign apps.
Furthermore, Malwares tend to repeatedly try to conduct malicious actions such as sending sensitive information and loading malicious codes because most of the time the servers of attackers are offline or shut down to escape detection.
% Thus, creating malwares to make sure that attackers receive sensitive data results in the unique traffic patterns of malwares.
Thus, creating malwares to ensure that malicious actions succeed results in the unique traffic patterns of malwares.
% Thus, the unique traffic patterns of malwares results from creating malwares to make sure that attackers receive sensitive data .
That scheme can detect malwares which communicate with other networks to conduct malicious actions by using the unique traffic patterns of malwares.
The traffic patterns based features regarding four categories, namely DNS, HTTP, TCP and origin-destination based ones are extracted from traffic data of an app running on a real device.
Finally, these features are fed into machine learning classifier such as Decision Tree and Random Forest for detection.
% The previous scheme can deal with the malwares which send encrypted data because the traffic based features are extracted without deep packets inspection.


\section{Shortcoming of the previous scheme } 
Although the features of the previous scheme can cope with encrypted traffic data, traffic pattern based features are subject to usage situations of apps.
Hence, since attackers can disguise traffic patterns of malwares by creating malwares whose traffic patterns are similar to benign ones, the previous scheme cannot detect such malwares.
In particular, that scheme is not applicable to repackaged malwares. 
Because a repackaged malware is created by injecting malicious components into an original app, the traffic pattern of a repackaged malware is similar to that of an original app.
Thus, repackaged malwares can evade the previous scheme.
In order to resolve the shortcoming, it is necessary to propose the features which can deal with encrypted traffic data and are hard to be disguised by attackers.  
