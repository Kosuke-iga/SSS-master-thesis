\chapter{Previous Scheme}\markright{Previous Scheme}\label{sec:previous_scheme}
\section{Overview of the previous scheme} 
In order to deal with spreading IoT malwares with device-side detections in more practical means, CNN is utilized in the previous scheme \cite{previous}.
The main idea of the previous scheme is that there exists the difference of GSIs converted from each malware Binary Data (BD) which is always acquired in IoT devices due to the exectable format of them.
In that scheme \cite{previous}, a malware binary is reformatted as an 8-bit sequence and then be converted to a GSI which has one channel and pixel values from 0 to 255.
Some examples of malware and benign software GSIs are shown by ず挿入.
By comparison, some novel differences between them can be observed even by human eyes.
For example, it can be seen that malware GSIs always are more dense.
On the other hand, the GSI of benign softwares tend to have larger header parts than malwares.

These GSIs can represents each features and be fed into CNN.
Since, CNN can extract effective features from these GSI automatically by learning deep non-linear features even that can be hardly descovered and understood by human eyes.
Furthermore, once trained, the network itself is lightweight and can be run with tiny computational resources, since only the trained parameters and information of network structure are kept.
Thus, CNN is a suitable scheme for IoT malware detections since it does not demand feature engineering and also computational costs while the detection.

\section{Shortcoming of the previous scheme } 
Although the features of the previous scheme can cope with encrypted traffic data, traffic pattern based features are subject to usage situations of apps.
Hence, since attackers can disguise traffic patterns of malwares by creating malwares whose traffic patterns are similar to benign ones, the previous scheme cannot detect such malwares.
In particular, that scheme is not applicable to repackaged malwares. 
Because a repackaged malware is created by injecting malicious components into an original app, the traffic pattern of a repackaged malware is similar to that of an original app.
Thus, repackaged malwares can evade the previous scheme.
In order to resolve the shortcoming, it is necessary to propose the features which can deal with encrypted traffic data and are hard to be disguised by attackers.  
