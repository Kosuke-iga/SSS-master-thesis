\chapter{Introduction}\markright{Chapter 1: Introduction}\label{Sec:Introduction}

In these days, the Internet of Things (IoT) is interconnetcting a large number of electronic devices with a variety of applications in our lives, such as smart appliances, smart houses, smart grid, energy management systems, and so on, at a tremendous speed.
The number of the IoT devices is continue to increase.
It is estimated to be about 50 billion all over the world by 2030 \cite{}.
They become targets of malware attacks due to the rapid increase and the development into devices which have interconnectivity in these days.
Unfortunately, vulnerable IoT devices are spreading since the countermeasures cannot keep up with this trend of IoT malware attacks.
Thus, this circumstance results in the urgency of detecting IoT malwares.

%Existing practical solutions for detecting malwares are all static approaches since they are not suitable for dynamic approach due to limited computational resources of them.
Existing solutions for detecting malwares are mainly classified into router-side schemes \cite{} and device-side schemes \cite{}.

Router-side schemes pay attention to the fact that there exists strong evidences of malwares in network traffic by utilizing the features extracted from incoming packets of an scanning/infectiong phase such as destination IP address during a certain time, port number and so on, at a gateway of the network.
These schemes focus on the fact that adversaries have to scan and infect IoT devices before attacking itself and theses phases cause some features in the incoming packets.
Although router-side schemes are useful, these schemes are subject to internal attacks of the local network itself since the packets do not go through the router.

In order to cope with the limitations of router-side schemes, device-side schemes are proposed.
Device-side schemes are conducted by utilizing the features extracted from malware itself at each IoT devices.
Most of the device-side schemes utilze various features which can be extract statically since IoT devices are not suitable for dynamic approach due to thier limited computing and storage capabilities.
\cite{} utilezes the features regarding to control flow graphs acquired by manipulated analysis and definitions of the features.
Although \cite{} can extract represantative features of the application, it is a time-consuming and laborious method due to the analysis for an enormous number of IoT malwares.

In order to cope with the spreading IoT malwares with device-side detections, some researchers utilize Convolutional Neural Network (CNN) in their scheme.
\cite{} utilezes GrayScale Image (GSI) converted from raw bytes of each malware binary for its CNN.
Although it is suitable for IoT malware detection since CNN can extract malware features from GSI automatically without feature engineering, it can be avoided with evasive techniques by adversaries to binary files or GSIs of malware. 
%More and more malwares tend to interact with external networks by using encrypted traffic in order to hide malicious payloads.
%Thus, it is necessary to iextract useful features from malwares without deep packet inspection.
%That scheme leverages the fact that there exists the difference of network traffic patterns between benign apps and malwares.
%Network traffic pattern based features extracted from traffic data of running apps are fed into machine learning classifier, and it detects malwares.
%Although that shcme is a promising approach due to applicability to encrypted traffic data, traffic pattern based features are easy to be disguised by attackers.
%Hence, the malwares whose traffic patterns are similar to that of benign apps can evade that scheme.

In order to detect such malwares with exisiting two evasive techniques, the one is to binary files and the other is to GSIs, in this paper, we propose an evasion-resilient IoT malware detection scheme with invalidating adversarial byte sequences and 1D convolutional filters.
The main idea of our scheme is that evasive techniques by adverasaries 限られている 

The main idea of our scheme is that malwares tend to communicate with untrusted destination servers in order to transfer encrypted malicious payloads such as sensitive information and malicious codes.
Untrusted servers can be identified on the basis of the level of SSL server certificate.
In order to encrypt traffic data, attackers have to introduce SSL server certificates to their servers.
Attackers tend to use an untrusted certificate to encrypt malicious payloads in many cases since passing rigorous examination process is required to obtain a trusted certificate.
Thus, \our scheme can detect malwares by using SSL server certificate based features since their servers tend to be untrusted.
However, since benign apps also communicate with untrusted servers, the detection performance may be degraded.
In order to eliminate such situation, \we introduce required permission base weight values to SSL server certificate based features because malwares inevitably have to require the permissions regarding malicious actions.
By doing this, \our scheme can obtain the more exact features of malwares. 
The contributions of this paper are as follows: 

The rest of this paper is constructed as follows. 
After this introduction, related works are introduced in Section \ref{sec:related_work}.
The previous scheme and the shortcoming of that scheme are explained in Section \ref{sec:previous_scheme}.
Proposed scheme is described in Section \ref{sec:proposed_scheme}.
Various evaluation results are their interpretation are shown in Section \ref{sec:evaluation}.
Finally, the conclusions of this paper are presented in Section \ref{sec:conclusion}.

% \begin{enumerate}
  % \item Thtough an inspection regarding communication with destination servers of apps, \we discovered that there exists the difference of the level of SSL server certificates introduced to destination servers between benign apps and malwares.
  % \item In order to cope with recent trend of malwares which transfer encrypted data, \we propose a detection scheme using features based on destination servers and the level of SSL server certificate.
    % \Our scheme can detect malwares by utilizing the useful features that are applicable to encrypted traffic data and are hard to be disguised.
  % \item  \Our evaluation results show that \our scheme can improve detection performance in comparison to the previous scheme. 
    % Furthermore, we clearly demonstrate that \our scheme can deal with the malwares transferring encrypted malicious payloads and the shortcoming of the previous scheme.
% \end{enumerate} 
