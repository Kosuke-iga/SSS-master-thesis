\chapter{Conclusions}\markright{Chapter 6: Conclusions}\label{sec:conclusion}
In this paper, \we have proposed an Android malwares detection scheme using features based on destination servers and the level of SSL server certificate.
\We utilize SSL server certificate based features which are hard to be disguised.
In order to obtain more exact malicious features, \we introduce required permission based weight values.
% By utilizing \our features
The detection performance of \our scheme is better than the previous scheme.
\Our evaluation results show that \our scheme can deal with encrypted traffic data and the shortcoming of the previous scheme.
Furthermore, after manually analysizing misjudged benign apps, \we discover the malware which sends sensitive information to a untrusted server.
The analysis results demonstrate that \our scheme can detect malware that evade the detection of other antivirus scanners and the previous scheme.
% \we plan to excute apps on a real device to inspect the malwares which escape an emulator.

  % In this thesis, we have proposed a target link flooding attack prevention scheme by monitoring the increase of traceroute packets which is caused by the attacker's reconnaissance.
  % Our scheme collects traceroute packets passing through routers within an AS (Autonomous System) by detection servers located in an AS.
  % Our scheme learns the number of cumulative traceroute packets by AR model and calculates an abnormality of the number of traceroute packets.
  % From scores of the abnormality, our scheme detects the attack symptoms before a link congestion occurs.
  % Instead of characteristics of a congested link and the router's link failure detection mechanism, we use a behaviour of increase of traceroute packets which is caused by the attacker's reconnaissance.
  % Therefore our scheme easily distinguishes the target link flooding attack from normal link failures.
  % By computer simulations, we show that our scheme is effective at ASes which have many links and recieve many traceroute packets.
  % In the real Internet environment, our scheme is only effective on ASes whose number of links are more than 1,500.
