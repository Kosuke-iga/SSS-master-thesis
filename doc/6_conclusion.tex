\chapter{Conclusions}\markright{Chapter 6: Conclusions}\label{sec:conclusion}
In this thesis, \we have proposed an evasion-resilient IoT ,malware detection scheme with invalidating adversarial byte sequences and 1D convolutional filters. 
\We utilize remaining regions which contribute to decision making in both manipulated targets, GSI and BD, even after the attacks.
In order to emphasize the valuable regions in GSI, the defense method converting JP values to zero is proposed aginst AM attacks.
On the other hand, in order to emphasize the valuable regions in BD, the defense method showing the horizontal connection of appearing order of byte sequences with 1D Conv. is proposed aginst OM attacks.
The defense methods of \our scheme are effective in both cases.
In the evaluation of the method against AM, it improves detection accuracy without the loss of valuable information.
In the evaluation of the method against OM, it improves detection accuracy independent of the packing tool.
Furthermore, I discover the horizontal connections in encrypted byte regions and the efficiency of 1D Conv..

