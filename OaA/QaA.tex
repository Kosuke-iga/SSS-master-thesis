\documentclass[11pt, a4paper]{article}
\makeatletter

\def\id#1{\def\@id{#1}}
\def\department#1{\def\@department{#1}}
\def\@maketitle{
\begin{center}
  % \vspace{10mm}
  {\Large\bf \@title \par}% 論文のタイトル部分
  % \vspace{10mm}
  % {\Large \@date\par} % 提出年月日部分
  \vspace{3mm}
  { \@department 学籍番号 \@id  \@author}  % 所属部分
  % {\Large 学籍番号 \@id \par}  % 学籍番号部分
  % \vspace{10mm}
  % {\large \@author}% 氏名 
\end{center}
\par\vskip 1.5em
}
\usepackage[top=20truemm,bottom=30truemm,left=15truemm,right=15truemm]{geometry}
\makeatother
\title{修士論文審査会における質問に対する回答文}
% \date{\today}
\department{慶應義塾大学大学院 理工学研究科 開放環境科学専攻\\笹瀬研究室 修士2年 }
\id{82018147}
\author{五十嵐幸佑}

\begin{document}
\maketitle
修士論文審査会においていただきました御指摘,御助言につきまして,以下のように回答させていただきます.

\begin{enumerate} 
	\item \textbf{「Junk Pixel (JP) の特定はどのタイミングで行うのか?」(山中先生)というご質問に対して}\\ \\
	CNNへの入力前の各検体に対する前処理として,検体であるバイナリデータ(実行ファイル)に基づくヘッダの解析によるJPの特定,及びグレースケール画像(GSI)におけるノイズの除去という二つの処理を行います.
	ヘッダ解析は,GSI変換前のバイナリデータに着目した処理になります.
	この処理は,GSI変換時における実行時にソフトウェアの挙動に影響を与えないバイト列由来のピクセルであるJPの特定を目的とした処理になります.
		このとき,本研究ではCareyら\cite{ae}の研究に基づき,コンパイラによりソースコードに挿入される実行時にメモリに割り当てられないデバック情報や,リンク時に使用されるメタデータを,JPに一致するバイト列と定義しています.
	これらのデータの特定をヘッダに記されているアドレスやセクション等の情報を元に行います.
	
	そして,ノイズ除去処理でGSI変換後の検体における特定したJPに対してピクセル値の0に統一することで,攻撃者によるピクセル操作による回避を防ぎます.
	
	この二つの処理は容易に行うことが可能であり,本手法において計算量等大きな負荷を与える処理ではありません.
	また,上記のように,JPの特定については,グレースケール画像を元にした処理ではなく,各実行ファイルであるバイナリデータに由来した処理であるため,攻撃者によるJP操作による提案手法の回避は不可能であると考えております.
       
	貴重なご意見,ご質問,ありがとうございました.
  \vspace{10mm}

  \item \textbf{「攻撃者は,本提案手法の検知に利用しているコンパイラにより挿入される実行時の挙動とは無関係なバイト列を挿入しないことで本提案手法の検知を回避できるのでは?」(寺岡先生)というご指摘に対して}\\ \\
	寺岡先生の仰るように,攻撃者によるそのような操作は可能です.
	しかしながら,Adversarial Malware (AM) 作成の際に良性判定に寄与するノイズの付与が可能であるのは,それらの実行時の挙動とは無関係なバイト列に限られるため,そもそもAM攻撃による従来手法の検知回避が成立しない状態になります.
	以上の理由から,攻撃者がAM攻撃を行う際には必ず実行時の挙動とは無関係なバイト領域が存在しているという前提で本研究を行なっております.
    
	貴重なご意見,ご質問,ありがとうございました.
  \vspace{10mm}

  \item \textbf{「AMとObfuscated Malware (OM) に対する防御手法はそれぞれ独立したアイディアなのか?」(金子先生)というご指摘に対して}\\ \\
    	両攻撃に対して,”回避攻撃後も操作が困難なピクセル帯/バイト列が,識別に有用な情報を持つ領域として残存している”という共通のアイディアを元に各防御手法の提案に至りました.
	しかしながら,検知機構全体で見たときにそれら二つの防御手法は,金子先生の御指摘どおり,独立した手法となっています.

	貴重なご意見,ご質問,ありがとうございました.


  \item \textbf{「従来手法ではなぜ1次元でなく,2次元での畳み込みフィルタが採用されていたのか?」(岡本先生)というご指摘に対して}\\ \\
	従来手法では難読化されたマルウェアの検知についての検討はなされておらず,回避操作の受けてないマルウェアの検知という前提で研究がされておりました.
	回避操作のなされていないマルウェア画像については従来の二次元でのフィルタでもある程度バイト列の横の繋がりというのが特徴として得られていた(と考えられる)ため,フィルタの形状の考察には至っておりませんでした.
	そのため,通常の画像解析で一般的な2次元フィルタがそのまま用いられておりました.
    
	貴重なご意見,ご質問,ありがとうございました.
  \vspace{10mm}
\end{enumerate}



\end{document}
