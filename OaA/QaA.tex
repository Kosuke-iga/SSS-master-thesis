\documentclass[11pt, a4paper]{article}
\makeatletter

\def\id#1{\def\@id{#1}}
\def\department#1{\def\@department{#1}}
\def\@maketitle{
\begin{center}
  % \vspace{10mm}
  {\Large\bf \@title \par}% 論文のタイトル部分
  % \vspace{10mm}
  % {\Large \@date\par} % 提出年月日部分
  \vspace{3mm}
  { \@department 学籍番号 \@id  \@author}  % 所属部分
  % {\Large 学籍番号 \@id \par}  % 学籍番号部分
  % \vspace{10mm}
  % {\large \@author}% 氏名 
\end{center}
\par\vskip 1.5em
}
\usepackage[top=20truemm,bottom=30truemm,left=15truemm,right=15truemm]{geometry}
\makeatother
\title{修士論文審査会における質問に対する回答文}
% \date{\today}
\department{慶應義塾大学大学院 理工学研究科 開放環境科学専攻\\笹瀬研究室 修士2年 }
\id{81719437}
\author{加藤広野}

\begin{document}
\maketitle
修士論文審査会においていただきました御指摘,御助言につきまして,以下のように回答させていただきます.

\begin{enumerate} 
  \item \textbf{「アプリの審査を手動で行う方が確実に悪性アプリを検知できるのではないか?」(山中先生)というご指摘に対して}\\ \\
    山中先生のおっしゃる通り,手動で審査を行なった場合の方が悪性アプリを確実に検知することができる場合が多いと考えられます.
    しかしながら,世界シェア85\%を超えるAndroidにおけるアプリの審査は,公開申請されるアプリの数が膨大であり,多くの人件費や時間を要することから,機械的に行われています.
    したがって,手動で審査を行うことは非現実的であり,全ての悪性アプリの公開を防ぐことは困難状況にあるため,公開されてしまった悪性アプリを機械的に検知する方式が必要とされています.
    以上の理由から,機械的に悪性アプリの検知を行う方式の検討を行いました.

     貴重なご意見,ご質問,ありがとうございました.
  \vspace{10mm}

  \item \textbf{「提案方式はどこで適用することを想定しているのか?」(山中先生)というご指摘に対して}\\ \\
    提案方式のシステムモデルとしては,公開されたアプリに対して適用することを想定してます.
    上記の回答で述べた様にAndroidにおけるアプリ審査は,全ての悪性アプリを検知することは困難な状況にあるため,公開されてしまった悪性アプリを検知することを想定しています.

     貴重なご意見,ご質問,ありがとうございました.
  \vspace{10mm}

  \item \textbf{「パーミッションを基にした重み付けにおいて,1をプラスしているのはなぜか?」(大槻先生)というご指摘に対して}\\ \\
    重み付けにおいて,1をプラスしない場合,危険なパーミッションを持たないアプリは,パーミッションを基にした重みが0になるため,SSLサーバ証明書を基にした特徴の値に掛け合わせる際に計算結果が0になってしまいます.
    したがって,SSLサーバ証明書を基にした特徴が消えてしまうことを避けるため,重み付けにおいて,1をプラスしています.

    貴重なご意見,ご質問,ありがとうございました.
  \vspace{10mm}

  \item \textbf{「良性と悪性はどのように入手したのか?また,データセット内の良性アプリは本当に良性なのか?」(金子先生)というご指摘に対して}\\ \\
    金子先生の御指摘どおり,データセット内の全ての良性アプリが良性であるという保証はありません.
    したがって,修士論文の評価において,良性アプリに関する分析を行い,その結果について記述いたしました.

    貴重なご意見,ご質問,ありがとうございました.
\end{enumerate}



\end{document}
